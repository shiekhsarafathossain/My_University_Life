\documentclass[a4paper,12pt]{article}
\usepackage{amsmath,amsfonts,amssymb}
\usepackage{xcolor}
\usepackage{fancyhdr}
\usepackage{tcolorbox}
\usepackage{graphicx}
\usepackage{geometry}
\geometry{margin=1in}

% Set up header and footer
\pagestyle{fancy}
\fancyhf{}
\fancyhead[L]{\textbf{Computer Networks}}
\fancyhead[R]{\textbf{Bit Stuffing Example}}
\fancyfoot[C]{\thepage}

\begin{document}

\title{\textbf{Bit Stuffing of a Bit Stream}}
\author{Prepared by: Expert Note Maker}
\date{\today}
\maketitle

% Introduction
\section*{Introduction to Bit Stuffing}
Bit stuffing is a technique used in communication protocols to avoid confusion between data and control information. It ensures that the flag sequence used to mark the start and end of a frame (commonly \texttt{01111110} in HDLC) does not appear in the transmitted data.

To achieve this, whenever a sender encounters five consecutive 1's in the data stream, it automatically inserts a `0` bit after these 1's. This `0` bit is referred to as the \textit{stuffed bit}. On the receiver side, the extra `0` bit is removed to retrieve the original bit stream.

% Problem Statement
\section*{Problem Statement}
Given the following bit stream:
\begin{center}
    \texttt{0101011111101000000101111101}
\end{center}
We need to apply the bit stuffing process and indicate the positions of the stuffed bits.

% Solution
\section*{Solution}
The given bit stream is:

\begin{center}
\texttt{0101011111101000000101111101}
\end{center}

To apply bit stuffing, we follow these steps:

\begin{enumerate}
    \item Scan the bit stream.
    \item Insert a `0` after every sequence of five consecutive `1`s.
\end{enumerate}

Let’s highlight the five consecutive `1`s and the stuffed bits in the bit stream:

\begin{tcolorbox}[colframe=blue, colback=blue!5, title=\textbf{Bit Stuffing Process}]
Original bit stream: \texttt{010101111\textcolor{red}{111}01000000101111101}
\begin{itemize}
    \item Between positions 6 and 11, there is a sequence of \texttt{11111}.
    \item After this sequence, we \textbf{insert a `0`} to avoid confusion.
\end{itemize}
Thus, the stuffed bit stream becomes:
\[
\texttt{010101111110\textcolor{red}{0}1000000101111101}
\]
\end{tcolorbox}

% Final Answer
\section*{Final Answer}
After applying bit stuffing, the bit stream becomes:
\[
\boxed{\texttt{010101111110\textcolor{red}{0}1000000101111101}}
\]
The \textcolor{red}{\textbf{stuffed bit}} is highlighted in red.

\vspace{1cm}
\textit{Note:} The stuffed bit is added after every sequence of five consecutive `1`s to ensure that the control flags are not misinterpreted.

\section*{Conclusion}
Bit stuffing is a vital process for reliable communication protocols. By inserting a `0` after five consecutive `1`s, we ensure the integrity of data transmission.

\end{document}
