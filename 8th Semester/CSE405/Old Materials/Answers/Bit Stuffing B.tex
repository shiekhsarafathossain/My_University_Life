```latex
\documentclass{article}
\usepackage[utf8]{inputenc}
\usepackage{amsmath}
\usepackage{xcolor}
\usepackage{tikz}

\title{Computer Networks Quiz: Bit Stuffing Solution (Verified)}
\author{Claude}
\date{}

\begin{document}

\maketitle

\section*{Problem}
What would be the bit pattern of the following bit stream after "Bit Stuffing"? Please indicate the stuffed bit.

\begin{center}
\texttt{0101011111101000000101111101}
\end{center}

\section*{Solution}
In bit stuffing, a '0' bit is inserted after every sequence of five consecutive '1' bits. This is done to prevent the accidental occurrence of the flag sequence (usually 01111110) within the data.

Let's apply bit stuffing to the given bit stream:

\begin{center}
\tikz[baseline]{\node[anchor=base,draw=blue!30,rounded corners,inner sep=1ex]{
\texttt{0101011111\color{red}{\textbf{0}}101000000101111\color{red}{\textbf{0}}101}
};}
\end{center}

The \textcolor{red}{\textbf{red}} bits are the stuffed bits.

\section*{Step-by-Step Verification}
Let's verify the solution step by step:

1. Original bit stream: \texttt{0101011111101000000101111101}

2. Find the first sequence of five consecutive '1' bits:
   \texttt{01010\underline{11111}101000000101111101}
   
   Insert a '0' after this sequence:
   \texttt{0101011111\color{red}{\textbf{0}}101000000101111101}

3. Continue scanning and find the next sequence of five '1' bits:
   \texttt{0101011111\color{red}{\textbf{0}}1010000001\underline{11111}01}
   
   Insert another '0' after this sequence:
   \texttt{0101011111\color{red}{\textbf{0}}101000000101111\color{red}{\textbf{0}}101}

4. No more sequences of five consecutive '1' bits remain.

\section*{Final Result}
The bit pattern after bit stuffing is:

\begin{center}
\fbox{\texttt{0101011111\color{red}{\textbf{0}}101000000101111\color{red}{\textbf{0}}101}}
\end{center}

\textit{Note: The stuffed bits are highlighted in \textcolor{red}{red} for clarity.}

\end{document}
```