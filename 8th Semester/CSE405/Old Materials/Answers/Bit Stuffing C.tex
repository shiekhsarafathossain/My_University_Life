\documentclass[a4paper,12pt]{article}
\usepackage{amsmath,amssymb}
\usepackage{xcolor}
\usepackage{fancyhdr}
\usepackage{tcolorbox}
\usepackage{graphicx}
\usepackage{geometry}
\geometry{margin=1in}

% Set up header and footer
\pagestyle{fancy}
\fancyhf{}
\fancyhead[L]{Computer Networks}
\fancyhead[R]{Bit Stuffing Problem}
\fancyfoot[C]{Page \thepage}

\begin{document}

% Title
\begin{center}
    \LARGE \textbf{Bit Stuffing for a Given Bit Stream}
\end{center}

\noindent\textbf{Given Bit Stream:}
\[
\texttt{0101011111101000000101111101}
\]

\noindent\textbf{Procedure:}
\begin{enumerate}
    \item The technique of \textbf{bit stuffing} is used to prevent confusion with frame delimiters (e.g., \texttt{01111110}).
    \item In this process, a \textbf{0 is inserted} after any sequence of \textbf{five consecutive `1's} to avoid interpreting it as a control flag.
\end{enumerate}

\noindent\textbf{Steps:}
\begin{enumerate}
    \item Start scanning the bit stream from the left.
    \item Identify sequences of 5 consecutive `1's`.
    \item Insert a `0` after each such sequence.
\end{enumerate}

\noindent\textbf{Applying to the Given Stream:}

\begin{itemize}
    \item In the given bit stream, the sequence \texttt{11111} occurs between the 6th and 11th bits.
    \item Thus, a `0` is inserted after the 11th bit.
\end{itemize}

\begin{tcolorbox}[colback=blue!5, colframe=black, title=\textbf{Bit Stuffing Example}]
Original bit stream: \texttt{0101011111101000000101111101}

Stuffed bit stream: \texttt{010101111110\textcolor{red}{0}1000000101111101}
\end{tcolorbox}

\noindent The inserted bit is marked in \textcolor{red}{\textbf{red}}.

\noindent\textbf{Final Stuffed Bit Stream:}
\[
\boxed{\texttt{010101111110\textcolor{red}{0}1000000101111101}}
\]

\noindent \textbf{Conclusion:} After applying bit stuffing, a `0` is inserted after the sequence of five consecutive `1's` in the bit stream.

\end{document}
